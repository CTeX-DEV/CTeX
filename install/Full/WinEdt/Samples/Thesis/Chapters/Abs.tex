% -----------------------------------------------------------------------------
% -*-TeX-*- -*-Hard-*- Smart Wrapping
% -----------------------------------------------------------------------------
% Thesis Abstract -------------------------------------------------------------

\prefacesection{Abstract}


The existence of invariant subspaces for bounded linear operators acting on
an infinite--dimensional Hilbert space appears to be one of the most
difficult questions in the theory of linear transformations. The question is
known as the {\em invariant subspace problem}. Very few affirmative answers
are known regarding this problem. One of the most prominent ones is the
theorem on the existence of hyper--invariant subspaces for compact operators
due to V.I.\,Lomonosov.

\smallskip

The aim of this work is to generalize Lomonosov's techniques in order to
apply them to a wider class of not necessarily compact operators. We start by
establishing a connection between the existence of invariant subspaces and
density of what we define as the associated Lomonosov space in a certain
function space. On a Hilbert space approximation with Lomonosov functions
results in an extended version of Burnside's Theorem. An application of this
theorem to the algebra generated by an essentially self--adjoint operator $A$
yields the existence of vector states on the space of all polynomials
restricted to the essential spectrum of $A$. Finally, the invariant subspace
problem for compact perturbations of self--adjoint operators is translated
into an extreme problem and the solution is obtained upon differentiating
certain real--valued functions at their extreme.

\smallskip

The invariant subspace theorem for essentially self--adjoint operators acting
on an infinite--dimensional real Hilbert space is the main result of this
work and represents an extension of the known techniques in the theory of
invariant subspaces.

% -----------------------------------------------------------------------------
