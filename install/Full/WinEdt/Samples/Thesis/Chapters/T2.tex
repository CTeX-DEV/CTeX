% -----------------------------------------------------------------------------
% -*-TeX-*- -*-Hard-*- Smart Wrapping
% -----------------------------------------------------------------------------
\def\baselinestretch{1}

\chapter{An Extension of Burnside's Theorem}

\def\baselinestretch{1.66}


%%% ---------------------------------------------------------------------------

In this chapter we combine differentiability of the Hilbert norm with the
Schauder--Tychonoff Fixed Point Theorem to show that for every weakly closed
subalgebra $\A\neq\BH$, acting on a real or complex Hilbert space $\h$, there
exist nonzero vectors $f,g\in\h$ such that for every $A\in\A$:
\[ \abs{\RE\seq{A f,g}}\leq\essnorm{\RE A}\seq{f,g}. \]
This result generalizes an extension of Burnside's Theorem, recently obtained
by V.I.\,Lomonosov, using rather different techniques. The theory developed
in this chapter has an interesting application to the invariant subspace
problem for essentially self--adjoint operators which is given in the last
chapter.

%%% ---------------------------------------------------------------------------
\goodbreak
\section{Introduction}

In the first chapter we defined the {\em Lomonosov space} and gave a
constructive proof of the approximation theorem inspired by the well known
result of V.I.~Lomonosov~\cite{Lom73}. This theorem was then applied to
obtain a connection between the existence of invariant subspaces for the norm
dual of an algebra of bounded operators on a Banach space, and density of the
associated Lomonosov space in certain function spaces. These results cover
recent characterizations of the invariant subspace problem by
Y.A.~Abramovich, C.D.~Aliprantis, and O.~Burkinshaw in~\cite{AAB95}, who
obtained their results using the techniques introduced in~\cite{Lom91} and
further exploited in~\cite{dB93}.

\medskip

In this chapter we combine differentiability of the Hilbert norm with a
construction of the Lomonosov functions and the Schauder--Tychonoff Fixed
Point Theorem to establish a connection between the Lomonosov space and the
transitive algebra problem~\cite{RR73}.

\medskip

We start by briefly introducing a simplified Hilbert space terminology, that
is consistent with the first chapter, where the corresponding terms are
defined in more general, Banach space setting.

\smallskip

\begin{defn}
Suppose $\s$ is a bounded closed convex subset of a real or complex Hilbert
space $\h$, equipped with the relative weak topology. The set of all
continuous functions $f\colon\s\To\h$, where both spaces are equipped with
the weak topology, is denoted by $C(\s,\h)$. Similarly, $C(\s,[0,1])$ stands
for the set of all weakly--continuous functions $f\colon\s\To[0,1]$.
\end{defn}

\smallskip

\begin{rem}
Recall that a bounded closed convex subset $\s$ in a Hilbert space is weakly
compact. Observe also, that a bounded linear operator $A\in\BH$ is in
$C(\s,\h)$. Whenever we say that $A$ is in $C(\s,\h)$, we actually refer to
the restriction of the operator $A\in\BH$ to the subset $\s\subset\h$.
\end{rem}

\bigskip

\goodbreak

\begin{defn}
Let $\A$ be a subset of $C(\s,\h)$. The convex set $\cal{L}(\A) \subset
C(\s,\h)$, defined by
\[ \cal{L}(\A) = \set{ \sum_{k=1}^n \alpha_k A_k\,\,|\,\,\, A_k\in\A,\,
   \alpha_k\in C(\s,[0,1]) \text{ and } \sum_{k=1}^n\alpha_k \equiv 1;
    \,\, n < \infty} \]
is called the {\em Lomonosov space} associated with the set $\A$, and a
function $\Lambda\in\cal{L}(\A)$ is called a {\em Lomonosov function}.
\end{defn}

\medskip

\begin{defn}
Let $\W$ be a basic weak neighborhood of a vector $f\in\h$:
\begin{equation}\label{e:WeakBase}
 \W=\set{g\in\h\,\,|\,\,\,\abs{\seq{f-g,h_k}} < 1, ~~ h_k\in\h, ~~~~
    k=1,\ldots,n; \,\, n < \infty}.
\end{equation}
A continuous nonnegative function $\Gamma_\W\colon\h\To[0,1]$, defined by
\begin{equation}\label{e:CIF}
  \Gamma_\W(g)= \prod_{k=1}^n \max\set{0,1-\abs{\seq{f-g,h_k}}},
\end{equation}
is called a {\em continuous indicator function} of $\W$.
\end{defn}

\medskip

\begin{rem}
Clearly, $\Gamma_\W$ is a nonnegative weakly continuous function and
\[ \W = \set{g\in\h\,\,|\,\,\,\Gamma_\W(g)>0}. \]
\end{rem}

\bigskip

The following proposition and its corollary introduce the idea that will lead
to the main result of this chapter.

\medskip

\goodbreak
\newpage

\begin{prop} \label{p:wAPPROX}
Let $\s$ be a closed bounded and convex subset of $\h$. Suppose the set
$\A\subset{}C(\s,\h)$ satisfies the following property:
\begin{quote}
  {For every $s\in\s$ there exists a function $A_s\in\A$ together with a
  weak neighborhood $\W_s$ of $s$ such that $A_s(\W_s)\subset\s$.}
\end{quote}
Then there exists a Lomonosov function $\Lambda\in\cal{L}(\A)$ that maps the
set $\s$ into itself.
\end{prop}

\begin{proof}
By the hypothesis for every point $s\in\s$ there exists a function $A_s$
together with a basic weak neighborhood $\W_s$ of $s$ such that
$A_s(\W_s)\subset\s$. In this way we obtain an open cover of $\s$. Since $\s$
is a weakly compact set there exists a finite subcover $\W_1,\ldots,\W_n$,
together with functions $A_1,\ldots,A_n$, with the property that
$A_k(\W_k)\subset\s$ for $k=1,\ldots,n$.

Let $\Gamma_k:\s\To[0,1]$ denote the continuous indicator function of $\W_k$
as defined by~(\ref{e:CIF}). Each point $s\in\s$ lies at least in one
neighborhood $\W_k$ $(k=1,\ldots,n)$, therefore the sum
$\sum_{j=1}^n\Gamma_j(s)$ is strictly positive for all vectors $s\in\s$.
Hence, the functions $\alpha_k\colon\s\To[0,1]$, defined by
\[ \alpha_k(s) =
   \frac{\Gamma_k(s)}{\sum_{j=1}^n \Gamma_j(s)}
   \quad\quad   (k=1,\ldots,n), \]
are well defined and weakly continuous on $\s$. Also,
$\sum_{k=1}^n\alpha_k(s)=1$ for every $s\in\s$, and $\alpha_k(s)>0$ if and
only if $s\in\W_k$.

\smallskip

The Lomonosov function $\Lambda\colon\s\To\s$, in the Lomonosov space
$\cal{L}(\A)$, associated with the set of functions $\A\subset{}C(\s,\h)$, is
defined by
\[ \Lambda(s) = \sum_{k=1}^n \alpha_k(s) A_k(s). \]
Observe that $\Lambda(s)$ is a convex combination of the elements in $\s$,
and consequently, $\Lambda$ maps the set $\s$ into itself
(weak--continuously).
\end{proof}

\medskip

\goodbreak

\begin{cor} \label{c:wAPPROX}
Suppose $\A$ is a convex subset of $C(\s,\h)$ satisfying the condition of
Proposition~\ref{p:wAPPROX}. Then there exists an element $A\in\A$ with a
fixed point $s\in\s$.
\end{cor}

\begin{proof}
By the Schauder--Tychonoff Fixed Point Theorem the Lomonosov function
$\Lambda:\s\To\s$, constructed in the proof of Proposition~\ref{p:wAPPROX},
has a fixed point $s\in\s$. Let $A=\sum_{k=1}^n\alpha_k(s)A_k$. Convexity of
the set $\A$ implies that $A\in\A$. Furthermore, from $\Lambda(s)=s$ it
follows that $A(s)=s$.
\end{proof}

\begin{rem}
In our applications we will consider the situations when $\s$ is a closed
ball of radius $r\in(0,1)$ around a fixed unit vector $f_0\in\h$, and $\A$ is
a convex subset of $\BH$. If the set $\A$ satisfies the condition of
Proposition~\ref{p:wAPPROX} then Corollary~\ref{c:wAPPROX} implies that the
set $\A$ contains an operator $A$ with an eigenvalue $1$.

\medskip

This gives rise to the following two questions:

\smallskip

1.~ {\em When does the set $\A$ satisfy the condition of
Proposition~\ref{p:wAPPROX}?}

2.~ {\em When is the operator $A$ in Corollary~\ref{c:wAPPROX} different from
the identity operator? }

\smallskip

Complete continuity of compact operators, restricted to $\s$, yields an
affirmative answer to the first question whenever $\A$ is a set of compact
operators with the property that for every $s\in\s$ there exists an operator
$A_s\in\s$ such that $\norm{A_s s - f_0}<r$. Furthermore, if the space $\h$
is assumed to be infinite--dimensional then an affirmative answer to the
second question follows from the fact that the identity is not a compact
operator. However, compactness of the operators in $\A$ is much too strong an
assumption. In the next two sections we develop conditions based on the
properties of the essential spectrum and differentiability of the Hilbert
norm that will replace the condition of Proposition~\ref{p:wAPPROX}.
\end{rem}

%%% ---------------------------------------------------------------------------
\goodbreak
\section{On the Essential Spectrum}

In this section we state some well known properties of the essential spectrum
in the form applicable to the situations arising later. We start with a few
definitions and introduce notation and terminology that is consistent
throughout this chapter.

\begin{defn}
Suppose $\h$ is a real or complex Hilbert space. The algebra of all bounded
linear operators on $\h$ is denoted by $\BH$, while $\KH$ stands for the
ideal of compact operators in $\BH$. The {\em spectral radius} of the
operator $A\in\BH$ is denoted by $r(A)$ and its {\em essential norm}, i.e.
the norm of $A$ in the {\em Calkin algebra}\, $\BH/\KH$, is denoted by
$\essnorm{A}$.
\end{defn}

\begin{defn}
If $\lambda\in\Complex$ is a complex number then $\RE\lambda$ and
$\IM\lambda$ denote its real and imaginary parts respectively, i.e.
$\lambda=\RE\lambda+i\IM\lambda$. On the other hand, for a bounded linear
operator $A\in\BH$, $\RE{A}$ and $\IM{A}$ stand for its real and imaginary
parts:
\[ \RE{A}=\frac{A+A^*}{2} \quad \text{ and } \quad
   \IM{A}=\frac{A-A^*}{2}, \]
where $A^*$ is the Hilbert adjoint of $A$ in $\BH$.
\end{defn}

Clearly, for every $A\in\BH$ we have $A=\RE{A}+\IM{A}$. Furthermore, this
decomposition makes sense on a real or complex Hilbert space, and
\begin{equation}\label{e:EssIE}
  \essnorm{\RE A} \leq \essnorm{A} \leq \norm{A}.
\end{equation}

\begin{prop} \label{p:EssNR}
Suppose $\delta$ and $M$ are positive numbers, and $A$ is a fixed operator in
$\BH$. Then there exists a weak neighborhood $\W$ of $0$ in $\h$ such that
every vector $f\in\W$ with $\norm{f}\leq{}M$, satisfies the inequality
\[ \abs{\RE\seq{A f,f}} < \essnorm{\RE{}A}\norm{f}^2 + \delta. \]
\end{prop}

\begin{proof}
From $\RE\seq{A f,f}=\seq{(\RE{A})f,f}$ and $(\RE{A})=(\RE{A})^*$ it follows
that
\[ \norm{\RE{}A} = \sup_{f\neq0}
   \abs{\RE\seq{A f,f}}\norm{f}^{-2}. \]
By definition of the essential norm, we have
\[ \essnorm{\RE A} = \inf_{K\in\KH} \norm{(\RE A)+K} =
   \inf_{K\in\KH} \norm{\RE(A+K)}. \]
Hence, there exists a compact operator $K$ such that
\[ \essnorm{\RE A} > \norm{\RE(A+K)}    - \tfrac{1}{2}\delta M^{-2} \geq
   \abs{\RE\seq{(A+K)f,f}}\norm{f}^{-2} - \tfrac{1}{2}\delta M^{-2} . \]
The proposition now follows by the mixed (weak--to--norm) continuity of
compact operators on bounded sets.
\end{proof}

The following proposition plays an important role in the subsequent sections.

\begin{prop}\label{p:ESSPECT}
Suppose $\h$ is a real or complex Hilbert space, and $\lambda\in\Complex$ is
a point in the spectrum of the operator $A\in\BH$, such that
\begin{equation} \label{e:ESSINEQ}
 \abs{\RE\lambda} > \essnorm{\RE{}A}.
\end{equation}
Then the norm closure of the algebra generated by $A$ contains a nonzero
finite--rank operator.
\end{prop}

\begin{proof}
We may assume that the Hilbert space $\h$ is complex, as long as we can
construct a finite--rank operator in the closure of the {\em real} algebra
generated by $A$.

Clearly, (\ref{e:ESSINEQ}) implies that $\lambda$ is not in the essential
spectrum of $A$. From the well known properties of the essential spectrum
(for example, Theorem~6.8 and Proposition~6.9 in~\cite[p.~366]{Con90}), it
follows that every point in the spectrum of the operator $A$, satisfying the
condition~(\ref{e:ESSINEQ}) is an isolated eigenvalue of $A$, and the
corresponding Riesz projection has finite rank.

\goodbreak

After first replacing the operator $A$ by $-A$ in the case when
$\RE\lambda<0$, and then substituting the translation
$A-\max\set{\RE\lambda\,\,|\,\,\,\lambda\in\sigma(A)}$ for $A$, we may assume
that $\max_{\lambda\in\sigma(A)}{\RE\lambda}=0$. The condition
(\ref{e:ESSINEQ}) implies that
\[ \sigma_0(A)=\set{\lambda\in\sigma(A)\,\,|\,\,\,\RE\lambda=0}, \]
is a nonempty finite set, consisting of eigenvalues of $A$ with finite
multiplicity. By the Riesz Decomposition Theorem~\cite[p.~31]{RR73}, the
space $\h$ can be decomposed as $\h=\h_1\oplus\h_2$, where
$\dim(\h_1)<\infty$, and the operator $A$ is similar to the operator
$A_1\oplus{}A_2$, corresponding to this decomposition. Furthermore, the
spectrum of $A_1$ is $\sigma_0(A)$, and the spectrum of $A_2$ lies in the
open left complex half--plane. Therefore $r(e^{t{}A_2})<1$ for $t>0$, while
$r(e^{t{}A_1})=1$ for any real argument $t$. By Rota's
Theorem~\cite[p.~136]{Pau86}, the operator $e^{A_2}$ is similar to a strict
contraction. Consequently,
\[ \lim_{n\To\infty}\frac{ e^{n{}A_2} }{ \norm{e^{n{}A_1}} }=0. \]
On the other hand, finite--dimensionality of $\h_1$ implies that the sequence
\[ T_n = \frac{e^{n{}A_1}}{\norm{e^{n{}A_1}}}, ~
         \qquad n=0,1,\ldots, \]
has a subsequence converging in norm to a nonzero finite--rank operator.
\end{proof}

\begin{rem}
Recall that the exponential function $e^A$ admits the power series:
\[ e^A = \sum_{n=0}^{\infty} \frac{A^n}{n!}. \]
Hence, the finite--rank operator constructed in the proof of
Proposition~\ref{p:ESSPECT} is indeed contained in the norm closure of the
{\em real} algebra, generated by $A$.
\end{rem}

%%% ---------------------------------------------------------------------------
\goodbreak
\section{Preliminary Geometric Results}

{
\def\baselinestretch{1.6}

This section contains preliminary results that are needed in the constructive
proof of the main theorem, given in the next section. The results presented
here are mostly easy observations and the proofs are somehow tedious and
straightforward calculations, involving the standard ``$\eps,\delta$''
arguments.

\smallskip

Throughout this section we make the following conventions:

$\h$ is a real or complex Hilbert space. Fix a unit vector $f_0\in\h$ and
choose a positive number $r\in(0,1)$. The set $\s$ is defined as follows:
\[ \s = \set{f\in\h\,\,|\,\,\, \norm{f_0-f}\leq r}. \]

\begin{lem} \label{l:APPROX}
Let $\W$ be a subset of $\s$ and let $A$ be a bounded linear operator on
$\h$. Suppose that
\[ \RE\seq{A{}f,f_0-f}\geq\delta>0, \quad \text{\em for all $f\in\W$.} \]
Then there exists a positive number $\mu>0$, such that for any
$\eps\in(0,\mu)$:
\[ \norm{f_0 - (I+\eps A)f} < \norm{f_0 - f},
   \quad \text{\em for all $f\in\W$.} \]
\end{lem}

\begin{proof}
Note that $f\in\s$ implies $\norm{f} \leq 1 + r < 2$. Therefore for all
$f\in\s$:
\[ \norm{A{}f} \leq \norm{A}\norm{f} \leq (1+r) \norm{A} < 2\norm{A}. \]
Set $\mu=\tfrac{\delta}{2\norm{A}^2}$. For any $\eps\in(0,\mu)$ and $f\in\W$
we have:
\begin{equation*}
  \begin{split}
    \norm{f_0 - (I+\eps A)f}^2 &= \norm{f_0-f - \eps A f}^2 \\
     & = \norm{f_0-f}^2 - 2\eps\RE\seq{A{}f,f_0-f} + \eps^2\norm{A{}f}^2 \\
     & < \norm{f_0-f}^2 - 2\eps\delta + \eps^2 4\norm{A}^2 \\
     & = \norm{f_0-f}^2 - 2\eps(\delta - \eps 2\norm{A}^2) < \norm{f_0-f}^2.
  \end{split}
\end{equation*}
Hence $\mu$ is the required positive number.
\end{proof}

\def\baselinestretch{1.66}
}

\newpage

\begin{rem}
Let $\psi'(0)$ denote the derivative of the function
\[ \psi(t)=\norm{(I+t{A})f-f_0}^2, \]
with respect to $t$, at the point $t=0$. A straightforward calculation yields
\[ \psi'(0)=-2\RE\seq{A{f},f_0-f}. \]
Therefore the statement of Lemma~\ref{l:APPROX} corresponds to the well known
fact that a real function with (strictly) negative derivative is (strictly)
decreasing. Note, however, that positivity of $\RE\seq{A{f},f_0-f}$ does not
imply that the mapping $\Psi(f)=(I+\eps A)f$ is a contraction, as a function
from $\W$ to $\s$.
\end{rem}

\medskip

Lemma~\ref{l:APPROX} gives a numerical criterion for the subset $\W\subset\s$
to be mapped into $\s$, namely positivity of the function
$\Phi(f)=\RE\seq{A{}f,f_0-f}$ on $\W$. Since $\Phi(f_0)=0$, this criterion
cannot be employed at the point $f_0$. However, the problem of constructing a
function $\Lambda\colon\s\To\s$ can be easily reduced to the subset of $\s$
not containing the point $f_0$. A simple observation in $\Real^2$ suggests
the following definition.

\newcommand{\Polar}{{\cal P}_{\s}}
\begin{defn}
For a fixed ball $\s=\set{f\in\h\,\,|\,\,\, \norm{f_0-f}\leq r}$ around the
unit vector $f_0\in\h$, the {\em polar hyperplane} $\Polar$, of the origin
with respect to $\s$, is defined by the following set:
\[ \Polar = \set{f\in\h\,\,|\,\,\, \seq{f,f_0}=1-r^2}. \]
\end{defn}

\begin{rem}
Every vector $f$ in $\Polar\cap\s$ can be decomposed as\, $ f = (1-r^2)f_0 +
g$, where $g\perp f_0$ and $\norm{g}^2\leq r^2(1-r^2)$. In particular, the
boundary of $\Polar\cap\s$:
\[ \Polar \cap \partial \s = \set{(1-r^2)f_0 + g\,\,\big|\,\,\,
   g\perp f_0 \quad\text{ and }\quad \norm{g}^2=r^2(1-r^2)}, \]
contains exactly the points where the tangents from the origin to the ball
$\s$ intersect the set $\s$. Recall that in $\Real^2$ such a line is called a
{\em polar},\, and our definition is just a straightforward generalization of
this geometric term to the higher dimensional Hilbert spaces.
\end{rem}

\medskip

The following lemma will reduce the problem of constructing a Lomonosov
function $\Lambda\colon\s\To\s$ to the polar hyperplane.

\begin{lem} \label{l:SCALE}
The function $\Lambda_0\colon \s \To \s$, defined by
\[ \Lambda_0(f) = \frac{1}{r^2+\seq{f,f_0}} f, \]
maps the set $\s = \set{f\in\h\,\,|\,\,\, \norm{f-f_0}\leq r}$
weak--continuously into itself. Furthermore, the set of all fixed points for
$\Lambda_0$ is equal to $\Polar \cap \s$.
\end{lem}

\begin{proof}
Since $\RE\seq{f,f_0}>0$ for $f\in\s$, it follows that $\Lambda_0$ is well
defined and weakly continuous on $\s$. Clearly, $f\in\s$ is a fixed point for
$\Lambda_0$ if and only if $r^2 + \seq{f,f_0}=1$. By the definition of the
polar hyperplane, that is equivalent to $f\in\Polar\cap\s$.

We have to prove that $\norm{\Lambda_0(f)-f_0}\leq r$ for all $f\in\s$.

Every vector $f\in\s$ can be decomposed as $f=\seq{f,f_0}f_0+g$, where $g
\perp f_0$ and $\norm{g}^2\leq r^2 - \abs{1-\seq{f,f_0}}^2$.

A straightforward calculation, using this decomposition, yields:
\begin{equation*}
  \begin{split}
    \norm{\Lambda_0(f)-f_0}^2
    &= \norm{\frac{1}{r^2+\seq{f,f_0}} f - f_0}^2
        = \frac{1}{\abs{r^2+\seq{f,f_0}}^2} \left(r^4+\norm{g}^2\right)\\
    &\leq \frac{1}{\abs{r^2+\seq{f,f_0}}^2}
          \left(r^4+r^2-\abs{1-\seq{f,f_0}}^2\right).
  \end{split}
\end{equation*}
The conclusion follows if we can establish the following inequality:
\[ r^4+r^2 - \abs{1-\seq{f,f_0}}^2 \leq r^2\abs{r^2+\seq{f,f_0}}^2. \]
Setting $\seq{f,f_0}=x+i{}y$, this can be translated to
\[ r^4+r^2 - (1-x)^2 - y^2 \leq r^2(r^2 + x)^2 + r^2y^2, \]
or equivalently,
\begin{equation*}
  \begin{split}
   (1+r^2)y^2 &\geq r^4+r^2 - (1-x)^2 - r^2(r^2 + x)^2\\
              & =  (r^4+r^2-1-r^6) + 2(1-r^4)x - (1+r^2)x^2  \\
              & = -(1+r^2)\big((1-r^2) - x\big)^2.
  \end{split}
\end{equation*}
The last inequality is obviously always satisfied, with the strict inequality
holding everywhere, except in the polar hyperplane $\Polar$.
\end{proof}

\smallskip

\begin{defn}
For every operator $A$ in $\BH$ define a real function
$\Delta_A\colon\s\To\Real$ as follows:
\[ \Delta_A(f) = \frac{1}{r^2(1-r^2)}\RE\seq{A{}f, f_0-f}. \]
\end{defn}

\begin{rem}
Note that $\Delta_A$ is a ``normalization'' of the function $\RE\seq{A{}f,
f_0-f}$ in Lemma~\ref{l:APPROX}.

From the definition of the set $\Polar$ it follows that every vector $f$ in
$\Polar\cap\s$ can be decomposed as $f=(1-r^2)f_0 + r\sqrt{1-r^2} g$, where
$g \perp f_0$ and $\norm{g}\leq 1$. Consequently,
\begin{equation}\label{e:NUMERICAL}
  \begin{split}
   ~~ \quad ~~
   \Delta_A(f) &= \RE\seq{A\left(f_0+\tfrac{r}{\sqrt{1-r^2}}g\right),
                        f_0-\tfrac{\sqrt{1-r^2}}{r}g}\\
             &= \RE\seq{A f_0, f_0} - \RE\seq{A g, g} -
                \RE\seq{\left(\tfrac{\sqrt{1-r^2}}{r}A
                             -\tfrac{r}{\sqrt{1-r^2}}A^*\right) f_0, g}.
  \end{split}
\end{equation}
In particular, for the identity operator $I$ on $\h$, we have $\Delta_I(f) =
1 - \norm{g}^2$. Therefore, $\Delta_I\geq0$ on $\Polar\cap\s$, with the
equality $\Delta_I(f)=0$ holding if an only if $f\in\Polar\cap\partial\s$.

Observe that the function $\Delta_A\colon\s\To\Real$ is norm continuous, but
it is in general not weakly continuous, due to the presence of the quadratic
form $\RE\seq{A g,g} = \seq{(\RE{A})g, g}$ in~(\ref{e:NUMERICAL}).
\end{rem}

\smallskip

The next lemma imposes an additional condition on the operator $A$ that
guarantees the existence of a weak neighborhood of $f$ in $\s$ on which
$\Delta_A$ is positive.

\begin{lem} \label{l:ESS}
Suppose $f$ is a vector in the polar hyperplane $\Polar\cap\s$, satisfying
the following {\em strict} inequality for some $A \in \BH$:
\[ \Delta_A(f) > \essnorm{\RE{}A} \Delta_I(f). \]
Then there exists a positive number $\delta>0$, together with a weak
neighborhood $\W$ of $f$, such that for every $h\in\W\cap\s$:
\[ \Delta_A(h) > \essnorm{\RE{}A}\abs{\Delta_I(h)} + \delta. \]
\end{lem}

\begin{proof}
By the hypothesis, there exists a positive number $\delta>0$ such that:
\begin{equation}\label{e:EC1}
  \Delta_A(f) > \essnorm{\RE{}A}\Delta_I(f) + 5\delta.
\end{equation}
For any positive number $0<\eps<r^2$, define a weak neighborhood $\W_\eps$ of
$\Polar$:
\[ \W_\eps = \set{h\in\h\,\,|\,\,\,\abs{1-r^2-\seq{h,f_0}} < \eps}. \]
Every vector $h\in\W_\eps\cap\s$ can be decomposed as $h=\seq{h,f_0}f_0+g$,
where $g \perp f_0$ and
\[ \norm{g}^2 \leq r^2 - \abs{1-\seq{h,f_0}}^2 < r^2 - (r^2 -\eps)^2. \]
Estimating roughly, we conclude:
\begin{equation*}
  \begin{split}
   \Delta_I(h) &=
    \frac{\RE\big(\seq{h,f_0}(1-\overline{\seq{h,f_0}})\big)
         -\norm{g}^2}{r^2(1-r^2)}\\
    &>\frac{(1-r^2-\eps)(r^2-\eps)-r^2+(r^2-\eps)^2}{r^2(1-r^2)} >
    -\frac{3\eps}{r^2(1-r^2)}.
  \end{split}
\end{equation*}
Therefore, a weak neighborhood $\W_\eps$ of $\Polar$, such that
$\essnorm{\RE{}A} \Delta_I(h) > -\delta$, for every vector
$h\in\W_\eps\cap\s$, can be obtained by setting
\[ \eps=\frac{\delta{}r^2(1-r^2)}{1+3\essnorm{\RE{}A}}. \]

A straightforward calculation yields:
\begin{equation}\label{e:EC2} ~
  ~\Delta_A(f+g) = \Delta_A(f) + \tfrac{1}{r^2(1-r^2)}\left(
   \RE\seq{A g,f_0-f} - \RE\seq{A f,g} - \RE\seq{A g,g} \right).
\end{equation}
Proposition~\ref{p:EssNR} implies the existence of a weak neighborhood $\W_1$
of $0$, such that for every vector $g\in\W_1$, with $\norm{g}\leq2$:
\begin{equation}\label{e:EC3}
  \RE\seq{A g,g} < \essnorm{\RE{}A} \norm{g}^2 + r^2(1-r^2)\delta.
\end{equation}
Clearly, by the weak--continuity of both sides of the inequality, there
exists a weak neighborhood $\W_2$ of $0$, such that for $g\in\W_2$:
\begin{equation}\label{e:EC4}
  \begin{split}
    \RE\seq{A g,f_0-f} & - \RE\seq{A f,g} > \\
     & \essnorm{\RE{}A}\big(\RE\seq{g,f_0-f}-\RE\seq{f,g}\big)
       -r^2(1-r^2)\delta.
  \end{split}
\end{equation}
Let $\W = \W_\eps\cap\big(f + \W_1\cap\W_2\big)$ be a weak neighborhood of
$f$. Every vector $h$ in $\W\cap\s$ can be written as $h=f+g$, where $g \in
\W_1\cap\W_2$ and $\norm{g}<2$. Putting the inequalities
(\ref{e:EC1}\,--\,\ref{e:EC4}) together, and using $\essnorm{\RE{}A}
\Delta_I(h) > -\delta$, implies:
\begin{equation*}
  \begin{split}
   \Delta_A(h) &= \Delta_A(f+g)\\
   &= \Delta_A(f) + \tfrac{1}{r^2(1-r^2)}\left(
        \RE\seq{A g,f_0-f} - \RE\seq{A f,g} - \RE\seq{A g,g}\right)\\
   &> \essnorm{\RE{}A}\big( \Delta_I(f) + \tfrac{1}{r^2(1-r^2)}(
        \RE\seq{g,f_0-f} - \RE\seq{f,g} - \RE\seq{g,g})\big) + 3\delta\\
   &= \essnorm{\RE{}A} \Delta_I(f+g) + 3\delta\\
   &= \essnorm{\RE{}A} \Delta_I(h) + 3\delta\\
   &> \essnorm{\RE{}A} \abs{\Delta_I(h)} + \delta.
  \end{split}
\end{equation*}
Consequently,\, $\W$ is a weak neighborhood of $f$, with the required
property.
\end{proof}

%%% ---------------------------------------------------------------------------
\goodbreak
\section{The Main Result}

We are now ready to give the main result of this chapter, which is quite
technical, but applicable to several situations discussed later.

\smallskip

\begin{prop} \label{p:ESS}
Let $\A\subset\BH$ be a convex subset of bounded linear operators acting on a
real or complex Hilbert space $\h$. Fix a unit vector $f_0\in\h$ and choose a
positive number $r\in(0,1)$. Suppose that for every vector $g\perp{}f_0$ and
$\norm{g}\leq1$, there exists an operator $A\in\A$, satisfying the following
strict inequality:
\begin{equation}\label{e:ESSCON}
  \RE\seq{A\left(f_0+\tfrac{r}{\sqrt{1-r^2}}g\right),
                 f_0-\tfrac{\sqrt{1-r^2}}{r}g} >
  \essnorm{\RE{}A}(1 - \norm{g}^2).
\end{equation}
Then $\A$ contains an operator $A_0$, with an eigenvector in the set
\[ \s=\set{f\in\h\,\,|\,\,\,\norm{f_0-f}\leq{}r}, \]
and the corresponding eigenvalue $\lambda$ satisfies the condition:
$\abs{\RE\lambda}>\essnorm{\RE{}A_0}$.
\end{prop}

\begin{proof}
Introducing the polar hyperplane $\Polar$ as before, observe that by
(\ref{e:NUMERICAL}) the condition~(\ref{e:ESSCON}) implies that every vector
in $\Polar\cap\s$ satisfies the hypothesis of Lemma~\ref{l:ESS} for some
operator $A\in\A$. Consequently, for every vector $f$ in $\Polar\cap\s$ there
exists an operator $A\in\A$, together with a (basic) weak neighborhood $\W$
of $f$, and a positive number $\delta$, such that for every $h\in\W\cap\s$:
\[ \Delta_A(h) > \essnorm{\RE{}A}\abs{\Delta_I(h)} + \delta. \]
By Lemma~\ref{l:APPROX} there exists a positive number $\mu$ such that the
operator $I+\eps A$ maps the set $\W\cap\s$ into $\s$ whenever
$\eps\in(0,\mu)$.

\goodbreak
\def\baselinestretch{1.5}
\setlinespacing{1.5}

In this way we obtain a weakly open cover of $\Polar\cap\s$ with basic
neighborhoods. By the weak--compactness of the set $\Polar\cap\s$, there
exists a finite subcover $\W_1,\ldots,\W_n$, together with the operators
$A_k$ in $\A$, and positive numbers $\mu_k>0$, such that for
$\eps\in(0,\mu_k)$ the operator $I+\eps{}A_k$ maps the set $\W_k\cap\s$ into
$\s$, and
\begin{equation}\label{e:EINEQ}
  \Delta_{A_k}(h) > \essnorm{\RE{}A_k} \abs{\Delta_I(h)},
\end{equation}
for every $h\in\W_k\cap\s$.

Define the weakly open set $
\W_0=\set{f\in\h\,\,|\,\,\,\abs{\seq{f,f_0}-(1-r^2)}>0}$. Associated with the
set $\W_0$ is its continuous indicator function $\Gamma_0\colon\h\To\RPlus$:
\[ \Gamma_0(f)=\abs{\seq{f-f_0}-(1-r^2)}, \]
and the function $\Lambda_0\colon\s\To\s$, defined in Lemma~\ref{l:SCALE}:
$\Lambda_0(f) = \big(r^2+\seq{f,f_0}\big)^{-1}f$.

\smallskip

Fix a positive number $\eps\in(0,\min\set{\mu_1,\ldots,\mu_n})$, and recall
that every basic weak neighborhood $\W_k$ admits a continuous indicator
function $\Gamma_k\colon\s\To[0,1]$, defined by~(\ref{e:CIF}). Each point
$f\in\s$ lies at least in one neighborhood $\W_k$ $(k=0,\ldots,n)$, therefore
the sum $\sum_{j=0}^n\Gamma_j(f)$ is strictly positive for all vectors
$f\in\s$. Hence, the functions $\alpha_k\colon\s\To[0,1]$,
\[ \alpha_k(f) =
   \frac{\Gamma_k(f)}{\sum_{j=0}^n \Gamma_j(f)}
   \quad\quad   (k=0,\ldots,n), \]
are well defined and weakly continuous on $\s$. Also,
$\sum_{k=0}^n\alpha_k(f)=1$ for every $f \in \s$, and $\alpha_k(f)>0$ if and
only if $f\in\W_k$.

\smallskip

The Lomonosov function $\Lambda\colon\s\To\s$, in the Lomonosov space
$\cal{L}({\A\cup\Lambda_0})$, associated with the set of functions
$\A\cup\Lambda_0\subset{}C(\s,\h)$, is defined by
\[ \Lambda(f) = \frac{\alpha_0(f)}{r^2+\seq{f,f_0}} f +
   \sum_{k=1}^n \alpha_k(f) (I+\eps A_k)f. \]
Observe, that $\Lambda(f)$ is a convex combination of the elements in $\s$,
and consequently, $\Lambda$ maps the set $\s$ into itself
(weak--continuously).

\setlinespacing{1.5}
\def\baselinestretch{1.66}
\goodbreak

The Schauder--Tychonoff Fixed Point Theorem implies that the Lomonosov
function $\Lambda\colon\s\To\s$ has a fixed point $f_1\in\s$. From
$\Lambda(f_1)=f_1$, we conclude:
\begin{equation*}
  \begin{split}
     \eps \big(\sum_{k=1}^n \alpha_k(f_1) A_k\big)f_1 & =
     \big(1-\frac{\alpha_0(f_1)}{r^2+\seq{f_1,f_0}} -
     \sum_{k=1}^n \alpha_k(f_1)\big)f_1 \\
     & = \alpha_0(f_1)\big(1 -  \frac{1}{r^2+\seq{f_1,f_0}}\big)f_1.
  \end{split}
\end{equation*}
Outside the set $\W_1\cup\ldots\cup\W_n$ the function $\Lambda$ equals
$\Lambda_0$ and has no fixed points. Consequently, $f_1\in\W_k$ for at least
one index $k\in\set{1,\ldots,n}$, and $\sum_{j=1}^n\alpha_j(f_1)>0$. Set
\[ \beta_k=\frac{\alpha_k(f_1)}{\sum_{j=1}^n \alpha_j(f_1)}=
           \frac{\alpha_k(f_1)}{1-\alpha_0(f_1)},
           \qquad (k=1,\ldots,n). \]
Then $A_0=\sum_{k=1}^n \beta_k A_k$ is an operator in the convex set $\A$.
Clearly, $f_1\in\s$ is an eigenvector for $A_0$, corresponding to the
eigenvalue $\lambda$:
\begin{equation}\label{e:LAMBDA}
   \lambda = \frac{\alpha_0(f_1)}{\eps\big(1-\alpha_0(f_1)\big)}\big(1 -
   \frac{1}{r^2+\seq{f_1,f_0}}\big).
\end{equation}
Recall that by (\ref{e:EINEQ}) the strict inequality $\Delta_{A_k}(f_1) >
\essnorm{\RE{}A_k}\abs{\Delta_I(f_1)}$ is satisfied whenever
$\alpha_k(f_1)>0$ (or equivalently $\beta_k>0$). Therefore, nonnegativity of
the coefficients $\beta_k$ and subadditivity of the essential norm, imply
\begin{equation*}
   \Delta_{A_0}(f_1) =
   \sum_{k=1}^n \beta_k \Delta_{A_k}(f_1)
   > \sum_{k=1}^n \beta_k \essnorm{\RE{}A_k}\abs{\Delta_I(f_1)}
   \geq \essnorm{\RE{}A_0}\abs{\Delta_I(f_1)}.
\end{equation*}
By~(\ref{e:LAMBDA}) the sign of $\IM\lambda$ is the same as the sign of
$\IM\seq{f_1,f_0}=\IM\seq{f_1,f_0-f_1}$. Hence, from $A_0f_1=\lambda{}f_1$
and $\Delta_{A_0}(f_1)>\essnorm{\RE{}A_0}\abs{\Delta_I(f_1)}$, we conclude:
\begin{equation*}
  \begin{split}
     \abs{\RE\lambda}\abs{\Delta_I(f_1)} &\geq (\RE\lambda)\Delta_I(f_1) =
         \tfrac{1}{r^2(1-r^2)}\RE\lambda \RE\seq{f_1, f_0-f_1} \\
   &\geq \tfrac{1}{r^2(1-r^2)}\big(\RE\lambda\RE\seq{f_1, f_0-f_1}-
                                   \IM\lambda\IM\seq{f_1, f_0-f_1}\big) \\
   &= \tfrac{1}{r^2(1-r^2)}\RE\seq{\lambda f_1,f_0-f_1} = \Delta_{A_0}(f_1)
    > \essnorm{\RE{}A_0}\abs{\Delta_I(f_1)}.
  \end{split}
\end{equation*}
The strict inequality implies that $\Delta_I(f_1)\neq0$, and consequently
$\lambda$ satisfies the required condition:
$\abs{\RE\lambda}>\essnorm{\RE{}A_0}$.
\end{proof}

%%% ---------------------------------------------------------------------------
\goodbreak
\section{Burnside's Theorem Revisited}

V.I.\,Lomonosov~\cite{Lom91} established the following extension of
Burnside's Theorem to infinite--dimensional Banach spaces:

\begin{thm}[V.I.\,Lomonosov, 1991]\label{t:LOMBUR}
Suppose $X$ is a complex Banach space and let $\A$ be a weakly closed proper
subalgebra of $\mathbf{B}(X)$, $\A\neq\mathbf{B}(X)$. Then there exists
$x\in{X''}$ and $y\in{X'}$, $x\neq0$ and $y\neq0$, such that for every
$A\in\A$
\begin{equation}\label{e:LB}
 \abs{\seq{x,A'y}}\leq\essnorm{A}.
\end{equation}
\end{thm}

\medskip

The techniques introduced in the proof of this theorem, based on the argument
of the celebrated de~Branges' proof of the Stone--Weierstrass
Theorem~\cite{dB59}, received further attention in \cite{AAB95,dB93}.
Although in the Hilbert space case Theorem~\ref{t:LOMBUR} is equivalent to
another theorem, also given in \cite{Lom91}, we take a different point of
view and employ Proposition~\ref{p:ESS} to obtain a stronger extension of
Burnside's Theorem to infinite--dimensional Hilbert spaces.

\medskip

The condition (\ref{e:LB}) is equivalent to the existence of {\em unit}
elements $x\in{X''}$ and $y\in{X'}$, and a nonnegative constant $C$
(depending on $\A$), such that
\begin{equation}\label{e:LBC}
 \abs{\seq{x,A'y}}\leq C \essnorm{A}, \qquad \text{ for all $A\in\A$}.
\end{equation}

In general, the constant $C$ depends on the space $X$, and the algebra $\A$.
It is not clear that on every Banach space there exists an upper bound for
$C$, satisfying the condition (\ref{e:LBC}), with respect to all proper
weakly closed subalgebras of $\mathbf{B}(X)$. An example of such a space
would certainly be of some interest. On the other hand, an affirmative answer
to the Transitive Algebra Problem~\cite{RR73} is equivalent to $C=0$.

\medskip

At the moment we can provide no results concerning the estimates for the
constant $C$ in any infinite--dimensional Banach space, other than a Hilbert
space. The next theorem implies that on a complex Hilbert space the constant
$C$ is at most one.

\begin{thm}\label{t:UNIFBURN}
Suppose $\h$ is a complex Hilbert space and let $\A$ be a weakly closed
subalgebra of $\BH$, $\A\neq\BH$. Then there exist nonzero vectors
$f,h\in\h$, such that for all $A\in\A$:
\begin{equation}\label{e:ReESS}
  \abs{\RE\seq{A f,h}}\leq\essnorm{\RE A}\seq{f,h}.
\end{equation}
\end{thm}

\begin{proof}
Suppose not; then the hypothesis of Proposition~\ref{p:ESS} is satisfied for
every unit vector $f_0$ and any positive number $r\in(0,1)$. Consequently,
the algebra $\A$ contains an operator $A_0$ with an eigenvalue $\lambda$,
satisfying the condition:
\[ \abs{\RE\lambda}>\essnorm{\RE{A}}. \]
Proposition~\ref{p:ESSPECT} implies that the algebra $\A$ contains a nonzero
finite--rank operator. Therefore~\cite[Theorem~8.2]{RR73}, the (transitive)
algebra $\A$ is weakly dense in $\BH$, contradicting the assumption
$\A\neq\BH$.
\end{proof}

\begin{rem}
Note that (after arbitrary choosing the unit vector $f_0$ and then letting
$r\To0$) the argument in the proof of Theorem~\ref{t:UNIFBURN} shows that the
set of all vectors $f\in\h$ for which there exists a nonzero vector $g\in\h$
satisfying the condition (\ref{e:ReESS}) is {\em dense} in $\h$.
\end{rem}

\begin{cor}\label{c:UNIFBURN}
Suppose $\h$ is a complex Hilbert space and let $\A$ be a weakly closed
subalgebra of $\BH$, $\A\neq\BH$. Then there exist {\em unit} vectors
$f,h\in\h$, such that for all $A\in\A$:
\begin{equation}\label{e:AbsESS}
  \abs{\seq{A f,h}}\leq\essnorm{A}.
\end{equation}
\end{cor}

\begin{proof}
By Theorem~\ref{t:UNIFBURN} there exist unit vectors $f,h\in\h$, such that
for every $A\in\A$
\[ \abs{\RE{\seq{A f,h}}}\leq\essnorm{\RE{A}}. \]
Set
\[ \xi = \left\lbrace
           \begin{array}{c l}
             1 & \text{if $\seq{A{}f,h}=0$},\\
             \tfrac{\seq{A{}f,h}}{\abs{\seq{A{}f,h}}} & \text{otherwise}.
           \end{array}
         \right. \]
Then
\[ \abs{\seq{A f,h}} = \abs{\RE\seq{\xi A f,h}} \leq \essnorm{\RE(\xi A)}
   \leq \essnorm{\xi A} = \essnorm{A}, \]
and consequently, the condition (\ref{e:AbsESS}) is weaker than
(\ref{e:ReESS}).
\end{proof}

\smallskip

The following definition yields an alternative formulation of the extended
Burnside's Theorem.

\begin{defn}\label{d:ESSTRANS}
A vector $f\in\h$ is called {\em essentially cyclic} for an algebra
$\A\subset\BH$, if for every nonzero vector $h\in\h$ there exists an operator
$A\in\A$ such that
\begin{equation*}
  \RE{\seq{A f,h}} > \essnorm{\RE{A}}\norm{f}\norm{h}.
\end{equation*}
We say that a subalgebra $\A$ of $\BH$ is {\em essentially transitive} if
every nonzero vector is essentially cyclic for $\A$.
\end{defn}

\begin{rem}
Note that our definition of essentially transitive algebras does not coincide
with the definition in~\cite{Lom91}. In view of the discussion preceding
Theorem~\ref{t:UNIFBURN}, we required that $C$ is at most one in the
definition of essential transitivity, while the definition in~\cite{Lom91}
assumes no upper bound on $C$.

\smallskip

According to Definition~\ref{d:ESSTRANS}, every essentially cyclic vector
$f\in\h$ is also cyclic for $\A$, i.e. the orbit
$\set{A{f}\,\,|\,\,\,A\in\A}$ is dense in $\h$. Consequently, an essentially
transitive algebra is also transitive, as defined in~\cite{RR73}.
\end{rem}

\medskip

Theorem~\ref{t:UNIFBURN} can be restated as the following solution of the
``Essentially Transitive Algebra Problem''.

\begin{thm}\label{t:ESSBURN}
An essentially transitive algebra of operators acting on a complex Hilbert
space $\h$, is weakly dense in $\BH$.
\end{thm}

\begin{rem}
The reader may have noticed that, by
Propositions~\ref{p:ESS}~and~\ref{p:ESSPECT}, an essentially transitive
algebra $\A$, acting on a {\em real}\, Hilbert space, still contains a
finite--rank operator in its norm closure. However, in the case of a real
Hilbert space this is not enough in order to conclude that $\A$ is weakly
dense in $\BH$. A commutative algebra $\cal{J}$, generated by the matrix
\[ J= \left[
        \begin{array}{r r}
          0 & -1 \\
          1 &  0
        \end{array}
  \right], \]
is an example of a proper essentially transitive algebra acting on $\Real^2$.
The tensor product $\BH\otimes\cal{J}$ is an example of such an algebra
acting on $\h\oplus\h$. However, the existence of a nonzero finite--rank
operator in the closure of an essentially transitive algebra, implies the
following commutative version of Theorem~\ref{t:ESSBURN}, which holds on real
or complex infinite--dimensional Hilbert spaces.
\end{rem}

\medskip

\begin{thm}\label{t:COMBURN}
A commutative algebra $\A$, of operators acting on a real or complex
infinite--dimensional Hilbert space, is {\em never} essentially transitive.
\end{thm}

\begin{proof}
By Propositions~\ref{p:ESS}~and~\ref{p:ESSPECT} the (norm) closure of every
essentially transitive algebra contains a nonzero finite--rank operator $T$.
Since $T{}A=A{}T$ for every $A\in\A$, it follows that the range of $T$ is a
nontrivial (finite--dimensional) invariant subspace for $\A$, contradicting
the (essential) transitivity of $\A$.
\end{proof}

%%% ---------------------------------------------------------------------------
