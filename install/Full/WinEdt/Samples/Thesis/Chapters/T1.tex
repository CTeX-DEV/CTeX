% -----------------------------------------------------------------------------
% -*-TeX-*- -*-Hard-*- Smart Wrapping
% -----------------------------------------------------------------------------
\def\baselinestretch{1}

\chapter{The Space of Lomonosov Functions}

\def\baselinestretch{1.66}


%%% ---------------------------------------------------------------------------

In this chapter we give a constructive proof of an abstract approximation
theorem, inspired by the celebrated result of V.I.\,Lomonosov~\cite{Lom73}.
This theorem is applied to obtain an alternative proof of some recent
characterizations of the invariant subspace problem, given in \cite{AAB95}.
We also establish density of non--cyclic vectors for certain convex sets of
compact quasinilpotent operators, and conclude with a related open question.
In Chapter 2 we extend the techniques introduced in this chapter to
non--compact operators acting on a Hilbert space.

\smallskip

%%% ---------------------------------------------------------------------------
\goodbreak
\section{Introduction}

V.I.~Lomonosov in his paper \cite{Lom91} conjectured that the adjoint of a
bounded operator on a Banach space has a non--trivial closed invariant
subspace. In view of the known examples of operators without an invariant
subspace \cite{Enf87,Rea85}, this is the strongest version of the invariant
subspace problem that can possibly have an affirmative answer. In particular,
if the Lomonosov conjecture is true, then every operator on a reflexive
Banach space has a non--trivial invariant subspace.

\bigskip
\goodbreak

Considering the strong influence of Lomonosov's results on the theory of
invariant subspaces, it is not surprising that both the conjecture and the
techniques developed in the interesting paper \cite{Lom91} received further
attention. L.~de~Branges used this result to obtain a characterization of the
invariant subspace problem in terms of density of certain functions. This
stimulated another characterization of the invariant subspace problem given
by Y.A.~Abramovich, C.D.~Aliprantis, and O.~Burkinshaw ~in~\cite{AAB95}.
Section~1.4 presents a more detailed account of this work.

We take a slightly different approach. First we give a constructive proof of
the approximation theorem, inspired by the well known Lomonosov construction
used in \cite{Lom73,RR73}. This theorem is then applied to give an
alternative proof of the main result in \cite{AAB95}. Our proof applies to
both real and complex Banach spaces, while the original result was
established for complex Banach spaces only. The alternative proof somehow
explains the role of compact operators that appear in the characterizations
of the invariant subspace problem~\cite{AAB95}.

\medskip

One may notice that the weak*--compactness of the unit ball in dual Banach
spaces plays an important role in \cite{AAB95,dB59,dB93,Lom91}, as well as in
the applications given in this chapter. In other words, if the Lomonosov
conjecture is true, then the compactness of the unit ball, with respect to
the weak* topology, is likely to be an important ingredient of its proof.

\smallskip

In the last section we put this observation to the test. A straightforward
application of the approximation theorem obtained in Section~1.3, together
with the Schauder--Tychonoff Fixed Point Theorem, yields density of
non--cyclic vectors for the dual of a convex set of compact quasinilpotent
operators. We end with the open problem of obtaining a similar result for the
original set, rather than its dual.

\bigskip

This work is more or less self--contained and the notation and terminology
used in it is (supposed to be) standard. However, here are a few conventions
that hold throughout this chapter:

\bigskip

By an operator we always mean a bounded linear operator acting on a real or
complex Banach space. If $\A$ is a set of operators and $K$ is a fixed
operator then $\A{}K$ stands for the set $\set{A{}K\,\,|\,\,\,A\in\A}$.
Saying that a set of operators $\A$, acting on a Banach space $X$, admits an
invariant subspace, means that there exists a non--trivial closed subspace of
$X$ that is invariant under all operators in $\A$. The space of all linear
operators on a Banach space $X$ is denoted by $\mathbf{B}(X)$, while $C(S,X)$
stands for the set of all continuous functions $f\colon{}S\To{}X$. If $S$ is
a subset of a Banach space $X$, then in saying that a linear operator $A$ is
in $C(S,X)$, we actually refer to the restriction of the operator $A$ to the
set $S$.

\bigskip

%%% ---------------------------------------------------------------------------
\goodbreak

\def\baselinestretch{1.1}

\section{Reflexive Topological Spaces and Continuous Indicator Functions}

This section introduces some topological preliminaries that lead to a fairly
general treatment of the approximation theory in the next section, where an
important role is played by the partition of unity and the ``continuous
indicator functions'' associated with a basis for the topology on a compact
domain of certain functions. The existence of continuous indicator functions
can be characterized by a purely topological property of the underlying
space, which is defined as ``reflexivity'' of the topological space. In this
section we introduce both concepts and establish the connection between them.

\def\baselinestretch{1.66}

\bigskip
\goodbreak
\newpage

\begin{defn}
Let $S=(S,\tau)$ be a topological space and denote by $C(S,\Real)$ the space
of all continuous real--valued functions on $S$. A topological space $S$ is
called {\em reflexive} if the topology $\tau$ coincides with the weakest
topology $\tau_w$ on $S$ for which all the functions in $C(S,\Real)$ are
continuous.
\end{defn}

\begin{rem}
The reflexivity of topological spaces is not to be confused with the
corresponding concept of the reflexivity of Banach spaces. Indeed, we
conclude this section by showing that every subset of a locally convex space
is reflexive.
\end{rem}

\begin{prop}
Reflexivity is a hereditary property; \, i.e. a subspace $S$ of a reflexive
topological space $X$ is reflexive with the relative topology.
\end{prop}

\begin{proof}
Consider the restrictions of the functions in $C(X,\Real)$ to the subset $S$,
and observe that they induce the relative topology on $S$, whenever $X$ is
reflexive.
\end{proof}

\begin{defn}
Suppose $U$ is an open subset of a topological space $S$. A continuous
function $\Gamma\colon S \To \RPlus$ is called a {\em continuous indicator
function} of $U$ in $S$ if
\[ U = \set{s\in S \,\,|\,\,\, \Gamma(s) > 0}.  \]
\end{defn}

\begin{rem}
If $X$ is a metric space then every open ball
\[ U=U(x_0,r)=\set{x\in X \,\,|\,\,\, d(x,x_0) < r}, \]
admits a continuous indicator function $\Gamma_U\colon{X}\To\RPlus$, defined
by
\[ \Gamma_U(x) = \max\set{0, \, r - d(x,x_0) }. \]
Furthermore, suppose $f \in C(S,X)$. Then the open set $V=f^{-1}(U)\subset S$
``inherits'' an indicator function from $U$ by setting:
$\Gamma_V(s)=\Gamma_U(f(s))$.
\end{rem}

\smallskip
\goodbreak

This yields the following characterization of reflexivity.

\begin{prop} \label{p:REFLEX}
A topological space $S=(S,\tau)$ is reflexive if and only if there exists an
open basis $\cal B$ for the topology $\tau$ such that each set $V\in\cal B$
admits a continuous indicator function $\Gamma_V\colon{}S\To\RPlus$.
\end{prop}

\begin{proof}
By definition of reflexivity, the family
\[ \cal B_0 = \set{f^{-1}(U) \,\,|\,\,\, f \in C(S,\Real) \text { and }
   U=(a,b) \subset \Real } \]
is a sub--basis for the topology $\tau$ on a reflexive space $S$. Clearly,
\[ \Gamma_U(t) = \max\set{0, \, \tfrac{b-a}{2}-\abs{\tfrac{a+b}{2}-t}} \]
is a continuous indicator function of the open interval $U=(a,b)$ in $\Real$.
Consequently, $\Gamma_V(s) = \Gamma_U\big(f(s)\big)$ is a continuous
indicator function for the set $V=f^{-1}(U)$ in $S$. Let $V=V_1
\cap\ldots\cap V_n$ for $V_k\in\cal{B}_0$. A continuous indicator function of
$V$ can be defined by
\[ \Gamma_V(s) = \prod_{k=1}^n \Gamma_{V_k}(s). \]
Therefore, each set in a basis
\[ \cal B = \set{V_1 \cap\ldots\cap V_n \,\,|\,\,\, V_k\in\cal{B}_0;
   \,\, n < \infty}, \]
admits a continuous indicator function.

\smallskip

The other direction is trivial, because the continuous indicator functions
form a subset of $C(S,\Real)$.
\end{proof}

\smallskip

\begin{rem}
The argument in the proof of Proposition~\ref{p:REFLEX} shows that the space
$\Real$ can be replaced by any metric vector space over $\Real$ in the
definition of reflexivity. In particular, considering the complex valued
functions would not change the definition of reflexivity.
\end{rem}

\smallskip
\goodbreak

\begin{rem}
While an open set $U$ is uniquely determined by any of its continuous
indicator functions, the converse is of course not true. However,
Proposition~\ref{p:REFLEX} allows us to choose a basis $\cal B$, and a
corresponding family
\[ \Gamma_{\cal B}=\set{\Gamma_U\colon S\To \RPlus
           \,\,|\,\,\, U\in\cal B} \]
of continuous indicator functions associated with the basis $\cal B$ for the
topology on a reflexive topological space $S$. In that sense, the
correspondence between the elements of $\cal B$ and an associated family of
continuous indicator functions $\Gamma_{\cal B}$ can be established.
\end{rem}

\medskip

Although not all topological spaces are reflexive (consider for example the
topology of finite complements on any infinite set) the next proposition
shows that convex balanced neighborhoods in a locally convex space admit
continuous indicator functions, and consequently, all locally convex spaces
are reflexive.

\smallskip

\begin{prop}
Every locally convex vector space $X$ is reflexive (as a topological space).
\end{prop}

\begin{proof}
Suppose $\cal B$ is a base for the topology on $X$ consisting of open convex
balanced sets. Then for each $U\in\cal{B}$:
\[ U = \set{x \in X \,\,|\,\,\, \mu_U(x) < 1}, \]
where $\mu_U$ is the Minkowski functional of $U$. The function
\[ \Gamma_U(x) =\max\set{0, \, 1-\mu_U(x)} \]
is a continuous indicator function for $U$. By Proposition~\ref{p:REFLEX},
$X$ is reflexive.
\end{proof}

%%% ---------------------------------------------------------------------------
\goodbreak
\section{Lomonosov Functions}

The proof of the celebrated result of V.I. Lomonosov \cite{Lom73,RR73} was
based on the ingenious idea of defining a continuous function with compact
domain in a Banach space, assuming that certain local conditions are met. In
this section we generalize this idea in the form of an approximation theorem.
Since our construction was greatly inspired by the proof of Lomonosov's
Lemma~\cite{Lom73,RR73}, we suggest the following definition.

\begin{defn}
Let $\A \subset C(S,X)$ be a subset of the space of continuous functions from
a topological space $S$ to a locally convex space $X$. The convex subset
$\cal{L}(\A) \subset C(S,X)$, defined by
\[ \cal{L}(\A) = \set{ \sum_{k=1}^n \alpha_k A_k \,\,|\,\,\, A_k\in\A,
   \alpha_k\in C(S,[0,1]) \text{ and } \sum_{k=1}^n\alpha_k \equiv 1;
   \,\, n < \infty}. \]
is called the {\em Lomonosov space} associated with the set $\A$, and a
function $\Lambda \in \cal{L}(\A)$ is called a {\em Lomonosov function}.
\end{defn}

\medskip

Recall that the {\em uniform topology} on $C(S,X)$ is induced by the topology
on a linear space $X$. If $\cal{B}$ is a local basis for the topology on $X$
then the sets
\[ \widehat{U}=\set{f\in C(S,X) \,\,|\,\,\, f(S)\subset U\in\cal{B}} \]
define a local basis for the uniform topology on $C(S,X)$. If $X$ is a
locally convex space then so is $C(S,X)$. In particular, if $X$ is a Banach
space then $C(S,X)$ with the uniform topology is a Banach space, as well.

\medskip

We are now ready to give a construction of the Lomonosov function that
uniformly approximates a continuous function within a given neighborhood.

\smallskip

\begin{lem} \label{l:APPROXI}
Let $\A \subset C(S,X)$ be a subset of continuous functions from a reflexive
compact topological space $S$ to a locally convex space $X$. Fix an open
convex neighborhood $U$ of \, $0$ in $X$. Suppose $f\colon{}S\To{X}$ is a
continuous function that at each point of $S$ can be approximated within $U$
by some element of $\A$; \, i.e. for every point $s \in S$ there exists a
function $A_s \in\A$ such that $A_s(s)-f(s)\in{U}$. Then there exists a
finite subset $\set{A_1,\ldots,A_n}$ of $\A$, together with continuous
nonnegative functions $\alpha_k\colon{}S\To[0,1]$, such that
$\sum_{k=1}^n\alpha_k\equiv1$, and the Lomonosov function
$\Lambda\in\cal{L}(\A)$, defined by
\[ \Lambda(s) = \sum_{k=1}^n \alpha_k(s) A_k(s), \]
lies in the prescribed neighborhood $\widehat{U}$ of $f$ in $C(S,X)$; \, i.e.
$\Lambda(s)-f(s)\in{U}$ for every $s\in{S}$.
\end{lem}

\begin{proof}
By the hypothesis, for every point $s\in{S}$ there exists a function
$A_s\in\A$ such that $A_s(s)-f(s)\in{U}$. Continuity of the functions $f$ and
$A_s$ implies the existence of a (basic) neighborhood $W_s$ of $s$ in $S$
such that $A_s(w)-f(w)\in{U}$ for every $w\in{W_s}$. In this way we obtain an
open cover for $S$ with the sets $W_s$. Compactness of $S$ yields a finite
subcover: $W_{s_1}\cup\ldots\cup{W_{s_n}}\supset{S}$.

\smallskip

Each set $W_s$ is associated with a continuous indicator function
$\Gamma_{W_s}\colon S \To \RPlus$. Every point in $S$ lies in at least one
neighborhood $W_{s_k}$; therefore the sum $\sum_{j=1}^n \Gamma_{W_{s_j}}(s)$
is strictly positive for all elements $s \in S$. Consequently, the functions
$\alpha_k\colon S \To [0,1]$, defined by
\[ \alpha_k(s) =
   \frac{\Gamma_{W_{s_k}}(s)}{\sum_{j=1}^n \Gamma_{W_{s_j}}(s)}
   \quad\quad   (k=1,\ldots,n), \]
are well defined and continuous on $S$. Also, $\sum_{k=1}^n\alpha_k(s)=1$ for
every $s \in S$, and $\alpha_k(s)>0$ if and only if $s \in W_{s_k}$.
Therefore, $\alpha_k(s)>0$ implies that $A_{s_k}(s)-f(s)\in{U}$.

Set $A_k=A_{s_k}$ $(k=1,\ldots,n)$. Continuity of the functions
$\alpha_k\colon S\To[0,1]$ and $A_k\colon S \To X$ implies that the Lomonosov
function $\Lambda\in\cal{L}(\A)$, defined by
\[ \Lambda(s) = \sum_{k=1}^n \alpha_k(s) A_k(s), \]
is continuous. Observe that
\[ \Lambda(s) - f(s) = \sum_{k=1}^n \alpha_k(s)
       \big(A_k(s)-f(s)\big) \]
is a convex combination of the elements in $U$, because only those
coefficients $\alpha_k(s)$ for which $A_k(s)-f(s)\in{U}$ are nonzero. Since
$U$ is a convex set, it follows that the image of $\Lambda-f$ is contained in
$U$. In other words, $\Lambda$ lies in the prescribed neighborhood
$\widehat{U}$ of $f$ in $C(S,X)$.
\end{proof}

\smallskip

\begin{rem}
The proof of Lomonosov's Lemma~\cite{Lom73,RR73} introduces a special case of
the above construction: $S$ is a compact set in a Banach space $X$, defined
as the closure of the image of the unit ball around a fixed vector $x_0$,
under a given nonzero compact operator $K$. Furthermore, the vector $x_0$ is
chosen so that the set $S$ doesn't contain the zero vector; $\A$ is the
restriction to $S$ of an algebra of bounded linear operators on $X$ that
admits no invariant subspaces. Under the stated hypothesis a construction of
the function $\Lambda\colon{S}\To{X}$ is given such that
$\Lambda\in\cal{L}(\A{K})$ maps $S$ into the unit ball around $x_0$; or
equivalently, the constant function $f\equiv{x_0}$ can be approximated on $S$
within $1$ by the elements of $\cal{L}(\A{K})$. It is clear from the original
construction as well as from Theorem~\ref{t:APPROX} that in that case the set
$S$ can be mapped into an arbitrary small neighborhood of $x_0$; or
equivalently, the function $f\equiv{x_0}$ is in the closure of the space
$\cal{L}(\A{K})$.
\end{rem}

\medskip
\goodbreak

The following approximation theorem follows immediately from
Lemma~\ref{l:APPROXI}.

\smallskip

\begin{thm}\label{t:APPROX}
Let $\A \subset C(S,X)$ be a subset of continuous functions from a reflexive
compact topological space $S$ to a locally convex space $X$. Suppose that
$f\colon S \To X$ is a continuous function that at each point of $S$ can be
approximated by some element of $\A$; \, i.e. for every $s\in{S}$ and every
neighborhood $U$ of \, $0$ in $X$ there exists a function $A_s\in\A$ such
that $A_s(s)-f(s)\in{U}$. Then the function $f$ can be approximated uniformly
on $S$ by the elements of the associated Lomonosov space $\cal{L}(\A)$.
\end{thm}

\smallskip

In the next section we employ Theorem~\ref{t:APPROX} to obtain an alternative
proof of a characterization of the existence of invariant subspaces for
algebras of bounded linear operators acting on a real or complex Banach
space. The complex version of this theorem was first established in
\cite{AAB95}, using rather different techniques built on the result of
L.\,de~Branges~\cite{dB93}.

%%% ---------------------------------------------------------------------------
\goodbreak

\def\baselinestretch{1.1}

\section{A Characterization of the Invariant Subspace Problem}

We introduce some basic concepts and notation that is consistent with
\cite{AAB95}. However, for more details and further references on the {\em
invariant subspace problem}, the reader is advised to consult the nicely
written and comprehensible original~\cite{AAB95}.

\def\baselinestretch{1.66}
\medskip

In this section $X$ stands for a real or complex Banach space of dimension
greater than one and $X'$ for its norm dual. The algebra of all bounded
linear operators on $X$ is denoted by $\mathbf{B}(X)$. If $\A$ is any subset
of $\mathbf{B}(X)$, then the adjoint set $\A'$ of $\A$ is defined by
$\A'=\set{A'\,\,|\,\,\, A\in\A}$, where $A'$ is the Banach adjoint of $A$.

\medskip
\goodbreak

The set $S=\set{x\in X'\,\,|\,\,\,\norm{x}\leq1}$ denotes the unit ball in
the dual space $X'$, equipped with its weak* topology.

\begin{defn}
The vector space of all continuous functions from $S$ to $X'$, where both
spaces are equipped with the weak* topology, is denoted by $C(S,X')$. As
usual, $C(S)$ denotes the commutative Banach algebra of all continuous
complex valued functions on $S$ with the uniform norm.
\end{defn}

Note that for each $T\in \mathbf{B}(X)$ the restriction of the adjoint
operator $T'\colon S\To X'$ is a member of $C(S,X')$. The vector space
$C(S,X')$, equipped with the norm
\[ \norm{f} = \sup_{s\in S} \norm{f(s)}, \]
is a Banach space.

\smallskip

The Banach space $C(S,X')$ played the central role in
\cite{AAB95,dB93,Lom91}. Lomonosov~\cite{Lom91} based his proof of an
interesting extension of Burnside's Theorem on the characterization of the
extreme points of the unit ball in the norm dual of $C(S,X')$ using the
argument of the celebrated de~Branges' proof of the Stone--Weierstrass
Theorem~\cite{dB59}. Louis~de~Branges~\cite{dB93} performed a deep analysis
of the behaviour of these extreme points that yielded a vector generalization
of the Weierstrass approximation theorem, similar to the approximation
theorem in the previous section. This approach resulted in a characterization
of the existence of a nontrivial invariant subspace for the algebra $\A'$ in
terms of density of the linear span of the set
\[ \set{ \alpha A'\,\,|\,\,\, \alpha \in C(S) \text{ and } A \in \A}, \]
in the space of restrictions of the adjoint operators to $S$, with respect to
a topology in $C(S,X')$, introduced by L.\,de~Branges.

\smallskip

Building upon this work, Y.A.~Abramovich, C.D.~Aliprantis, and O.~Burkinshaw
in~\cite{AAB95}, obtained the following characterizations of the existence of
a non--trivial invariant subspace for an algebra $\A$ of bounded linear
operators acting on a complex Banach space $X$:

\smallskip

\begin{thm}[Y.A.~Abramovich, C.D.~Aliprantis, and O.~Burkinshaw]
\label{t:AAB1} There is a non--trivial closed $\A$--invariant subspace of $X$
if and only if there exists an operator $T\in \mathbf{B}(X)$ and a compact
operator $K\in \mathbf{B}(X)$ such that $K'T'$ does not belong to the norm
closure of the vector subspace of $C(S,X')$ generated by the collection
\[ \set{\alpha K'A'\,\,|\,\,\, \alpha \in C(S) \text{ and } A \in \A}. \]
\end{thm}

\smallskip

\begin{thm}[Y.A.~Abramovich, C.D.~Aliprantis, and O.~Burkinshaw]
\label{t:AAB2} There is a non--trivial closed $\A'$--invariant subspace of
$X'$ if and only if there exists an operator $T\in \mathbf{B}(X)$ and a
compact operator $K\in \mathbf{B}(X)$ such that $T'K'$ does not belong to the
norm closure of the vector subspace of $C(S,X')$ generated by the collection
\[ \set{\alpha A'K'\,\,|\,\,\, \alpha \in C(S) \text{ and } A \in \A}. \]
\end{thm}

\medskip

We will give a short proof of both theorems as an application of
Theorem~\ref{t:APPROX}. Our proof applies to real or complex Banach spaces,
where in the case of a real Banach space, $C(S)$ stands for the Banach
algebra of all real--valued continuous functions on the set $S$.

\smallskip

Observe that the Lomonosov spaces $\cal{L}(K'\A')$ and $\cal{L}(\A'K')$, as
defined in the previous section, are subsets of the linear manifolds
introduced in Theorems~\ref{t:AAB1}~and~\ref{t:AAB2}.

\medskip

\begin{defn}
The vector $x$ in a Banach space $X$ is {\em cyclic} for the set of operators
$\A\subset \mathbf{B}(X)$ whenever the orbit
\[ \A x = \set{Ax\,\,|\,\,\, A \in \A} \]
is a dense subset of $X$. If every nonzero vector is cyclic for $\A$, we say
that $\A$ acts {\em transitively} on $X$. The terms {\em $\tau$--cyclic} and
{\em $\tau$--transitive} are defined in the same way, by considering the
space $X$ equipped with a topology $\tau$, instead of the norm.
\end{defn}

\medskip

The following well known characterizations of the existence of a non--trivial
invariant subspace for an algebra $\A \subset \mathbf{B}(X)$ follow
immediately from the definition.

\begin{prop} \label{p:ISC}
Suppose $\A\subset \mathbf{B}(X)$ is a subalgebra of bounded linear operators
on $X$. The following are equivalent:
  \begin{enumerate}
    \item $\A$ admits no nontrivial closed invariant subspace.
    \item $\A$ acts weak--transitively on $X$.
    \item $\A$ acts transitively on $X$.
    \item $\A'$ admits no nontrivial weak*--closed invariant subspace.
    \item $\A'$ acts weak*--transitively on $X'$.
  \end{enumerate}
\end{prop}

\medskip

As in \cite{AAB95} we introduce the subspace of completely continuous
functions in $C(S,X')$.

\begin{defn}
A function $f\in C(S,X')$ is said to be {\em completely continuous} if it is
continuous with respect to the weak* topology on $S$ and the norm topology on
$X'$. The subspace of all completely continuous functions is denoted by $\cal
K(S,X')$.
\end{defn}

Note that $K'\colon S \To X'$ is completely continuous whenever
$K\in\mathbf{B}(X)$ is a compact operator on $X$
(Theorem~6~\cite[p.\,486]{DS57}).

\medskip

We are now ready to give a short proof of
Theorems~\ref{t:AAB1}~and~\ref{t:AAB2}.

\medskip
\goodbreak

\noindent{\em Proof of Theorems~\ref{t:AAB1}~and~\ref{t:AAB2}. }

\bigskip

We start with Theorem~\ref{t:AAB2}, which is an almost straightforward
consequence of Proposition~\ref{p:ISC} and Theorem~\ref{t:APPROX}, applied to
the space $\cal K(S,X')$.

\medskip

Suppose $\A'$ has a non--trivial closed  invariant subspace. Then by
Proposition~\ref{p:ISC}, there exists a pair of nonzero vectors $x',y' \in S$
such that $\norm{A'x' - y'} \geq \eps > 0$ for all $A'\in\A'$. Choose any
vector $x \in X$ such that $\seq{x',x}=1$, and define the rank--one operators
$K=x\otimes x'$ and $T=x\otimes y'$. Clearly $T'K'x' = y'$, and since $T'K'$
cannot be approximated by the operators $A'K'$ at the point $x'$, it follows
that $T'K'$ is not in the norm closure of the linear space generated by
$\set{ \alpha A' K'\,\,|\,\,\, \alpha \in C(S) \text{ and } A \in \A}$.

\smallskip

Conversely, suppose $\A'$ admits no non--trivial closed invariant subspaces.
Therefore, $\A'$ acts transitively on $X'$, and consequently, every operator
$T'K'$ can be approximated by $A'K'$ at each point of $S$. Furthermore, since
$K$ is a compact operator in $\mathbf{B}(X)$, it follows that $T'K' \in \cal
K(S,X')$. Theorem~\ref{t:APPROX} implies that $T'K'$ is in the norm closure
of the Lomonosov space $\cal{L}(\A'K')$ and thus completes the proof.

\bigskip

The proof of Theorem~\ref{t:AAB1} is just slightly more complicated.

\medskip

Suppose the algebra $\A$ admits a nontrivial closed invariant subspace $\cal
M$. Then $\cal M^\perp$ is an invariant subspace for $\A'$. Fix a nonzero
vector $x \in \cal M$ and a nonzero functional $y'\in\cal{}M^\perp$, and
choose a vector $y \in X$ such that $\seq{y',y}=1$ and a functional $x'\in
X'$, with $\seq{x',x}=1$. Define the rank--one operators $K=x\otimes y'$ and
$T=y \otimes x'$. Then $K'T'y'=y'\neq0$, while $K' A' y' = 0$ for every $A'
\in \A'$. Consequently, the operator $K'T'$ is not in the norm closure of the
linear span of the completely continuous functions $\set{ \alpha K' A'
\,\,|\,\,\, \alpha \in C(S) \text{ and } A \in \A}$.

\smallskip

Conversely, suppose that there exists a compact operator $K$ and an operator
$T$ such that $K'T'$ is not in the closure of the linear subspace generated
by the completely continuous functions $\set{ \alpha K' A' \,\,|\,\,\, \alpha
\in C(S) \text{ and } A \in \A}$. Theorem~\ref{t:APPROX} implies that there
exists a nonzero vector $x'\in{}S$ such that the orbit
$\cal{M}=\set{K'A'x'\,\,|\,\,\, A\in\A}$ is not a norm--dense manifold in the
closure of the subspace $\cal{N}=\set{K'T'x'\,\,|\,\,\, T\in\mathbf{B}(X)}$.
By the Hahn--Banach Theorem there exists a functional $y''\in X''$ such that
$\seq{y'',K'A'x'}=0$ for every $A'\in\A'$, and $\seq{y'',K'T'x'}=1$ for some
$T \in \mathbf{B}(X)$. Consequently, $K''y''\neq0$. Compactness of $K$
implies that $y=K''y''\in X$, where $X$ is considered naturally embedded in
its second dual $X''$ (Theorem~5.5~\cite[p.\,185]{Con90} or
Theorem~2~\cite[p.\,482]{DS57}). From $\seq{x',A{}y}=0$ for all $A\in\A$, it
follows that the algebra $\A$ admits a non--trivial closed invariant
subspace. \qed

\medskip

It is possible to obtain similar characterizations that do not involve
compact operators, by considering some other topology on $C(S,X')$.
Theorem~3.1 in \cite{AAB95} and Theorem~6 in \cite{dB93} are examples of
results in that direction. We conclude this section by giving yet another
characterization of transitivity for an algebra $\A$ in terms of the closure
of the Lomonosov space $\cal{L}(\A')$ with respect to the {\em uniform}\,
topology $\tau_{w^*}$, induced on $C(S,X')$ by the weak* topology on the dual
Banach space $X'$.

\begin{thm} \label{t:ISAC}
Suppose $\A\subset{}\mathbf{B}(X)$ is a set of bounded linear operators on
$X$. Then the dual set $\A'=\set{A'\,\,|\,\,\,A\in\A}$ acts
weak*--transitively on $S$ if and only if the $\tau_{w^*}$--closure of the
Lomonosov space $\cal{L}(\A')$ is equal to the subspace
\[ C_0(S,X') = \set{f\in C(S,X')\,\,|\,\,\, f(0) = 0 }. \]
\end{thm}

\begin{proof}
The proof is almost identical to those of
Theorems~\ref{t:AAB1}~and~\ref{t:AAB2} except that Theorem~\ref{t:APPROX} is
now applied to the space $C(S,X')$ equipped with the topology $\tau_{w^*}$,
instead of ${\cal K}(S,X')$ with the norm topology.

If the set $\A'$ does not act weak*--transitively on $X'$ then there exists a
nonzero vector $x' \in S$ together with a weak* neighborhood $W$ of $y'$ in
$S$ such that $A' x' \not\in W$ for all $A'\in \A'$. Choose a vector $x \in
X$ such that $\seq{x',x}=1$ and let $T=x\otimes{}y'$. Then $T' x' = y'$, and
since $T'\in C_0(S,X')$ cannot be approximated by the operators in $\A'$ at
the point $x'$, it follows that $T'$ is not in the $\tau_{w^*}$--closure of
the associated Lomonosov space $\cal{L}(\A')$.

Conversely, if the set $\A'$ acts weak*--transitively on $S$ it follows from
Theorem~\ref{t:APPROX} that every function $f\in{}C_0(S,X')$ can be uniformly
approximated by the elements of $\cal{L}(\A')$, and thus $f$ is in the
$\tau_{w^*}$--closure of the Lomonosov space $\cal{L}(\A')$.
\end{proof}

\begin{cor} \label{c:ISYAC}
The algebra $\A$ admits no non--trivial closed invariant subspace if and only
if the $\tau_{w^*}$--closure of the Lomonosov space $\cal{L}(\A')$ is equal
to the subspace
\[ C_0(S,X') = \set{f\in C(S,X')\,\,|\,\,\, f(0) = 0 }. \]
\end{cor}

\begin{proof}
By Proposition~\ref{p:ISC}, the fact that $\A$ admits no non--trivial
invariant subspace is equivalent to $\A'$ acting weak*--transitively on $S$.
The result now follows from Theorem~\ref{t:ISAC}.
\end{proof}

\medskip

Note that the $\tau_{w^*}$--closure of the Lomonosov space
$\cal{L}(\mathbf{B}(X)')$ is always equal to $C_0(S,X')$. This observation
yields a few alternative formulations of Corollary~\ref{c:ISYAC}, which are
left to the reader.

%%% ---------------------------------------------------------------------------
\goodbreak

\def\baselinestretch{1}

\section{On Convex Sets of Compact Quasinilpotent Operators}

In this section we combine Lemma~\ref{l:APPROXI} with the Schauder--Tychonoff
Fixed Point Theorem, to establish a density result for non--cyclic vectors
for the dual of a convex set of compact quasinilpotent operators. We discuss
in what sense this result generalizes the celebrated Lomonosov
Lemma~\cite{Lom73}, and conclude with a problem of establishing a similar
result for the original set, rather than its dual.

\def\baselinestretch{1.66}
\smallskip

Recall that an operator is called {\em quasinilpotent} if $0$ is the only
point in its spectrum.

\begin{thm}\label{t:QC}
Suppose $\A$ is a convex set of compact quasinilpotent operators acting on a
real or complex Banach space $X$, and let $\A'=\set{A'\,\,|\,\,\,A\in\A}$ be
its dual in $\mathbf{B}(X')$. Then the set of non--cyclic vectors for $\A'$
is dense in $X'$.
\end{thm}

\begin{proof}
Suppose not; then there exists a functional $x_0\in{X'}$ and a positive
number $r > 0$ such that all vectors in the ball $S = \set{x \in X'\,|\,\,
\norm{x-x_0} \leq r}$ are cyclic for $\A'$. In particular, for every
functional $x \in S$ there exists an operator $A' \in \A'$ such that
$\norm{A'x - x_0} < r$. By Lemma~\ref{l:APPROXI} it follows that there exists
a Lomonosov function $\Lambda\in\cal{L}(\A')$ such that
$\norm{\Lambda(x)-x_0}<r$ for all $x\in{}S$. Consequently, $\Lambda$ maps $S$
into itself (weak*--continuously).

The Schauder--Tychonoff Fixed Point Theorem~\cite[p.\,456]{DS57} implies that
$\Lambda$ has a fixed point $x_1=\Lambda(x_1)$ in $S$. By the definition of
the Lomonosov space
\[ \Lambda = \sum_{k=1}^n \alpha_k A'_k,
   \,\,\, \text{ where } A_k\in\A, \,
   \alpha_k\in C(S,[0,1]) \text{ and } \sum_{k=1}^n\alpha_k \equiv 1;
   \,\, n < \infty . \]
Therefore $A'=\sum_{k=1}^n \alpha_k(x_1) A'_k$ is an operator in the convex
set $\A'$. From $\Lambda(x_1)=x_1$, we conclude that $A'x_1=x_1$. Since
$x_1\neq0$, it follows that $1$ is an eigenvalue for $A'$, contradicting the
assumption that $A'$ is a quasinilpotent operator.
\end{proof}

\goodbreak

\begin{rem}
Note that (unless $\A$ is assumed to be an algebra) it is not enough to
require that the operators in $\A'$ have no common invariant subspace, in
order to ensure that $\A'$ acts transitively on $X'$. It is indeed possible
to give examples of manifolds of nilpotent operators without a non--trivial
closed common invariant subspace. For such examples on finite--dimensional
vector spaces see \cite{MOR91}. By Theorem~\ref{t:QC} a manifold of such
operators cannot act transitively on the underlying space.
\end{rem}

Theorem~\ref{t:QC} does not follow from the original work of
V.I.\,Lomonosov~\cite{Lom73}. On the other hand, Lomonosov's
Lemma~\cite{Lom73} easily follows from Theorem~\ref{t:QC}, in the case when
the underlying Banach space is reflexive. In that sense Theorem~\ref{t:QC} is
a generalization of the Lomonosov Lemma.

{\nobreak This discussion suggests the following question, which we have not
been able to resolve: \nobreak
\begin{quote}
  {\em Does there exist a convex set $\A$ of compact quasinilpotent
  operators acting on a real or complex Banach space $X$ such that the set
  of non--cyclic vectors for $\A$ is {\em not} dense in $X$? }
\end{quote}
}

\medskip

By Theorem~\ref{t:QC} the underlying Banach space in such an example (if it
exists) cannot be reflexive. Furthermore, Lomonosov's Lemma implies that the
set $\A$ cannot be of the form $\A K$ or $K \A$, where $K$ is a fixed compact
operator. In particular, the set $\A$ in such an example can never be an
algebra.

\medskip

Since, according to Theorems~\ref{t:AAB1}~and~\ref{t:AAB2}, compact operators
are closely related to the existence of invariant subspaces for algebras of
operators, the answer to the above question might be of some interest to the
theory of invariant subspaces.

%%% ---------------------------------------------------------------------------
